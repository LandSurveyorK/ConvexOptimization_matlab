
\documentclass{article}     % Specifies the document class
\usepackage{amsmath}
\usepackage{amsthm}
\usepackage{mathtools}
\usepackage{amssymb}
\usepackage[utf8]{inputenc}
\usepackage[english]{babel}
\newtheorem{theorem}{Theorem}[section]
\newtheorem{corollary}{Corollary}[theorem]
\newtheorem{lemma}[theorem]{Lemma}
\newtheorem*{remark}{Remark}
\usepackage{enumerate}
\usepackage[linguistics]{forest}
\usepackage{float}


                             % The preamble begins here.
\title{Problem Sheet 3}  % Declares the document's title.
\author{WEI PENG}      % Declares the author's name.
\date{\today}      % Deleting this command produces today's date.

\newcommand{\ip}[2]{(#1, #2)}
                             % Defines \ip{arg1}{arg2} to mean
                             % (arg1, arg2).

%\newcommand{\ip}[2]{\langle #1 | #2\rangle}
                             % This is an alternative definition of
                             % \ip that is commented out.

\begin{document}           % End of preamble and beginning of text.
\maketitle                   % Produces the title.

\emph{You are incourage to discuss with your classmates whenever you find it helpful}

\noindent
\textbf{Probelm 1}. Let $B_t$ be a d-demensional Brownian motion.
\begin{enumerate}[(1)]
    \item Show that $X_t\triangleq O\cdot B_t$ and $Y_t\triangleq<\mu,B_t>$ are both Brownian motions, where $O$ is a $d\times d$ orthogonal matrix (i.e.$ O^TO=I_d$) and $\mu$ is a unit vector in $\mathbb{R}^d$.
    \item Given $s<u<t$, compute $E[B_{u}|B_s,B_t]$.
\end{enumerate}


\begin{proof}
\begin{enumerate}[(1)]
\item We only need to prove the third condition of Brownian motion.
\[O\cdot B_t-O\cdot B_s=O\cdot(B_t-B_s)\sim N(0,(t-s)OI_dO^T)=N(0,(t-s)I_d)\]
\[<u,B_t>-<u,B_s>=u^T(B_t-B_s)\sim N(0, (t-s)uI_du^T)=N(0,t-s)\]
\item  lemma: suppose $\sigma(\sigma(X),\mathcal{G})$ and $\mathcal{H}$ are 
independent, then \[E[X|\mathcal{G},\mathcal{H}] =E[X|\mathcal{G}]\]
 \begin{eqnarray*}
  E[B_u|B_s,B_t] & = & B_s+E[B_u-B_s|B_s,B_t]\\ 
    & = & B_s+E[B_u-B_s|B_t-B_s,B_s]\\ 
    & = & B_s+ E[B_u- B_s|B_t-B_s]\\ 
    & = & \frac{t-u}{t-s}B_s+\frac{u-s}{t-s}B_t
    \end{eqnarray*}
\end{enumerate}
\end{proof}

\noindent
\textbf{Probelm 2}. Let $B_t$ be a one dimensional Brownian motion.
\begin{enumerate}[(1)]
    \item Show that 
    \[X_t \triangleq \begin{cases}
        tB_{\frac{1}{t}}, & t>0\\ 
        0, & t=0
        \end{cases}\]
is a Brownian motion.
    \item Show that with probability one, there exist two sequences of positive times $s_n\downarrow 0$, $t_n\downarrow 0$. such that $B_{s_n}<0$ and $B_{t_n}>0$ for every $n$.
    \item Show that with probability one, $B$ is not differential at $t=0$, and hence conclude that with probability one, $t\mapsto B_t(\omega)$ is almost everywhere ono-differentiable.
\end{enumerate}
\begin{proof}
\begin{enumerate}[(1)]
    \item Only need to show that $X_t-X_s\sim N(0,t-s)$ and $X_t$ is continuous at $0$.
    actually,
   \[X_t-X_s = (t-s)B_{\frac{1}{t}}-s(B_{\frac{1}{s}}-B_{\frac{1}{t}})\sim N(0,t-s)\]
   By strong law of large number, $\frac{B_n}{n}\rightarrow 0$, as $n\rightarrow \infty$. For any $u$, there exists an $n$, such that $n\leq u < n+1$, applying Dood's inequality on $|B_u-B_n|$, which is a submartingale.
   \[P(\sup_{n\leq u<n+1}|B_u-B_n|>n^{\frac{2}{3}})\leq n^{-\frac{4}{3}}E[|B_u-B_n|^2]=n^{-\frac{4}{3}}\]
  by Borel-Cantelli lemma, $P(\sup_{n\leq u<n+1}|B_u-B_n|
  \leq n^{\frac{2}{3}}~ i.o.)=1$
  Rigorously, then there exists an $N$, when $n\geq N$, 
  \[P(\sup_{n\leq u<n+1}|\frac{B_u}{n}|\leq |\frac{B_n}{n}|+n^{-\frac{1}{3}})\geq P(\sup_{n\leq u<n+1}|B_u-B_n|\leq n^{\frac{2}{3}})=1\]
  thus, almost surely, $\lim_u \frac{B_u}{n}\rightarrow 0$, which implies $\lim_u \frac{B_u}{u}\rightarrow 0$.
 
   \item since for a Brownian motion, $P(\sup_{t\geq 0}X_t=\infty)=1$, $P(\inf_{t\geq 0}X_t=-\infty)=1$, there exists $S_n(\omega)\uparrow \infty$, $T_n(\omega)\uparrow \infty$, such that $X_{S_n}>0, X_{T_n}<0$. since $X_t=tB_{\frac{1}{t}}$, let $s_n=\frac{1}{S_n}\downarrow 0, t_n=\frac{1}{T_n}\downarrow 0$, such that 
   \[B_{s_n}>0,\quad B_{t_n}<0\]
   \item \[\lim_{t\rightarrow 0}\frac{1}{t}X_t=\lim_{t\rightarrow \infty}B_{t},\quad not~exists.\]
   since  $P(\sup_{t\geq 0}X_t=\infty)=1$, $P(\inf_{t\geq 0}X_t=-\infty)=1$. For other point $t$, note: $B_{t+s}-B_t=B_s$, hence conclude that with probability one, $t\rightarrow B_t(\omega)$ is almost everywhere non-differentiable.
\end{enumerate}
\end{proof}


\noindent
\textbf{Probelm 3}. Let $B_t$ be an $\{\mathcal{F}_t\}$-Brownian motion defined over a filtered probability space $(\Omega,\mathcal{F},\mathbb{P},\{\mathcal{F}_t\})$ where $\mathcal{F}_0$ contains all $\mathbb{P}$-null ses. Let $\sigma,\tau$ be two finite $\{\mathcal{F}_t\}$-stopping times such that $\sigma\leq \tau$. Show that 
\[E[f(B_{\tau})|\mathcal{F}_{\sigma}]=P_tf(x)|_{t=\tau-\sigma, x=B_{\sigma}}\]
Is it true that $B_{\tau}-B_{\sigma}$ and $\mathcal{F}_{\sigma}$ are independent.
\begin{proof}
Since $M_t(\theta)\triangleq e^{i<\theta,B_t>+\frac{1}{2}|\theta|^2t}, ~t\geq 0$ is a martingale. By applying O.S.T. on $(\sigma \wedge N, \tau \wedge N)$, we have 
\[E[e^{i<\theta,B_{\tau \wedge N}>+\frac{1}{2}|\theta|^2\tau \wedge N}|\mathcal{F}_{\sigma}] = E [  e^{i<\theta,B_{\sigma \wedge N}> +\frac{1}{2}|\theta|^2 \sigma \wedge N}]\]
since $\tau \in \mathcal{F}_{\sigma}$, and $\sigma$, $\tau$ are two finite stopping times.
\[E[e^{i<\theta, B_{\tau}-B_{\sigma}>}|\mathcal{F}_{\sigma}]=e^{ -\frac{1}{2} |\theta|^2 (\tau-\sigma) } \]
therefore $B_{\tau}-B_{\sigma}|\mathcal{F}_{\sigma}\sim N(0,\tau-\sigma)$, then 
\[E[f(B_{\tau})|\mathcal{F}_{\sigma}]=E[f(B_{\tau}-B_{\sigma}+B_{\sigma})|\mathcal{F}_{\sigma}]=P_t f(x)|_{t=\tau-\sigma,x=B_{\sigma}}\]
$B_{\tau}-B_{\sigma}$ and $\mathcal{F}_{\sigma}$ are not independent. Consider: $A\in \mathcal{F}_{\sigma}$, $0<P(A)<1$
\[\tau=\begin{cases}
 \sigma+t, &\omega\in A \\
 \sigma+s , & \omega \in A^c
 \end{cases}\]
 then $B_{\tau}-B_{\sigma}=(B_{\sigma+t}-B_{\sigma})1_A+(B_{\sigma+s}-B_{\sigma})1_{A^c}\sim N(0,t)1_A+N(0,s)1_{A^c}$
\end{proof}


\noindent
\textbf{Probelm 4}. 
 Let $B_t$ be a one dimensional Brownian motion with $\{\mathcal{F}^B_t\}$ being its augmented natural filtration. Construct an $\{\mathcal{F}^B_t\}$-stopping time $\tau$ explicitly which satisfies the Skorokhod embedding theorem for the unifrom distribution on the set $\{-2,-1,0,1,2\}$. Draw a picture to illustrate the construction as well.
 \begin{proof}
\begin{table}[htb]
\caption{Skorokhod Embedding}{
\centering
\begin{minipage}{0.45\textwidth}
\begin{tabular}{|c|c|c|c|c|c|}\hline
$D_1$ & -1 & -1 &  1 & 1 & 1\\\hline
$D_2$ & -1 & 1 & -1 & 1 & 1 \\\hline
$D_3$ & 1 & 1 &1 & -1 & 1\\\hline
\end{tabular}
\end{minipage}
\hfil
\begin{minipage}{0.45\textwidth}
\begin{tabular}{|c|c|c|c|c|c|}       \hline
$X$ & -2 & -1 & 0 & 1 &2 \\ \hline
$X_1$ & $-\frac{3}{2}$ & -$\frac{3}{2}$ & 1 &1 &1   \\\hline
$X_2$ &-2 & -1 & 0 & $\frac{3}{2}$ & $\frac{3}{2}$   \\\hline
$X_3$ & -2 & -1 & 0 & 1 & 2\\\hline
\end{tabular}
\end{minipage}}
\end{table}
\begin{forest}
 [0(1) [ -$\frac{3}{2}$ ($\frac{2}{5}$)
             [-1($\frac{1}{5}$)] [-2($\frac{1}{5}$)] 
       ]
       [1($\frac{3}{5}$) 
              [0($\frac{1}{5}$)]
              [$\frac{3}{2}$($\frac{2}{5}$)
                              [1($\frac{1}{5}$)][2($\frac{1}{5}$)]
              ]
        ]
]
\end{forest}
 
\end{proof}
Let $\tau_1=\inf\{ t\geq 0, B_t]\notin (-\frac{3}{2}, 1)\}$, 
\[\tau_2=\begin{cases}
\inf\{t\geq \tau_1, B_t\notin (-2,-1) \}, & \text{if}~ $B_{\tau_1}=-\frac{3}{2}\\
\inf\{t\geq \tau_1, B_t\notin (0,\frac{3}{2}), &\text{if}~ B_{\tau_1}=1
\end{cases}\]
\[\tau_3=\begin{cases}
\inf\{t\geq \tau_2, B_t\neq -2\}, & \text{if}~ B_{\tau_2}=-2\\
\inf\{t\geq \tau_2, B_t\neq -1\}, & \text{if}~ B_{\tau_2}=-1\\
\inf\{t\geq \tau_2, B_t\neq 0\}, & \text{if}~ B_{\tau_2}=0\\
\inf\{t\geq \tau_2, B_t\notin(1,2)\}, & \text{if}~ B_{\tau_2}=\frac{3}{2}
\end{cases}
\]


\noindent
\textbf{Probelm 5}.  Let $B_t$ be a 2-dimensional brownian motion starting at $i=(0,1)$ with $\{\mathcal{F}_t^B\}$ beging its augmented natural filtration.
\begin{enumerate}[(i)]
    \item Show that for each $\lambda \in \mathbb{R}^1$, the process $X_t^{\lambda}\triangleq e^{\lambda_{i}\cdot B_t}$ is an $\{\mathcal{F}_t^B\}$-martingale, where the multiplication in the exponential function is the complex multiplication.
    \item Let $\tau$ be the hitting time of the real axis by $B_t$, show that $B_{\tau}\in \mathbb{R}^1$ is Cauchy distributed (i.e. $\mathbb{P}(B_{\tau}\in dx)=(\pi(1+x^2))^{-1}, x\in \mathbb{R}^1$)
\end{enumerate}

\begin{proof}
\begin{enumerate}[(1)]
    \item 
    \begin{eqnarray*}
        E[X_t^{\lambda}|\mathcal{F}_s] & = & E[e^{\lambda(iB^1_t-B^2_t)}|\mathcal{F}_s]\\
         & = & e^{\lambda(iB^1_s-B^2_s)}E[e^{\lambda[i(B^1_t-B^1_s)-(B^2_t-B^2_s)]}]\\
         & = & e^{\lambda(iB^1_s-B^2_s)}\quad (i^2+1=0)\\
         & = & X_s^{\lambda}
         \end{eqnarray*}
     \item Let $\tau_0$ be the hitting time of the real axis by $B_t$ starting at $(0,1)$, $\tau_1$ be the hitting time of $y=1$ of $B_t$ starting at $(0,0)$
\end{enumerate}
\[\mathbb{P}^{(0,1)}(\tau_0\leq t)=\mathbb{P}^{(0,0)}(\tau_1\leq t)=\mathbb{P}^0(\tau_1\leq t)(\text{1-demension})=P(B_t\geq 1)\]
thus $\mathbb{P}(dt)=t^{-\frac{3}{2}}\cdot \frac{1}{\sqrt{2\pi}}\cdote^{-\frac{1}{2t}}$ and since $B_1$ and $B_2$ are independent
\begin{eqnarray*}
\mathbb{P}(dx)& = & \int_0^{\infty}\frac{1}{\sqrt{2\pi t}}e^{-\frac{x^2}{2t}}\cdot \frac{1}{\sqrt{2\pi}}t^{-\frac{3}{2}}e^{-\frac{1}{2t}}dt\\
     & = & \frac{1}{\pi(1+x^2)}
\end{eqnarray*}
\end{proof}



\noindent
\textbf{Probelm 6}. Let $X_t(\omega)\triangleq \omega_t$ be the coordinate process on $W^1$ and let $\{\mathcal{F}_t\}$ be the natural filtration of $X_t$. Denote $\mathbb{P}^{x,c}$ on $(W^1,\mathcal{B}(W^1))$ as the law of a one dimensional Brownian motion starting at $x$ with drift $c$.
\begin{enumerate}[(1)]
    \item Show that when restricted on each $\mathcal{F}_t$, $\mathbb{P}^{x,c}$ is absolutely continuous with respect to $\mathbb{P}^{x,0}$, with density given by 
    \[ \frac{d\mathbb{P}^{x,c}}{d\mathbb{P}^{x,0}}|=e^{c(X_t-x)-\frac{1}{2}c^2t}\]
    \item Define $S_t=\max_{0\leq sleq t}X_s$. Compute $\mathbb{P}^{0,c}(S_t\in dx, X_t\in dy)(x\geq 0,x\geq y)$.
\end{enumerate}

\begin{enumerate}
Consider $A=\{(X_{t_1}-x,X_{t_2}-X_{t_1},\cdots  X_{t_n}-X_{t_{n-1})\in B_1\times B_2\times \cdots \times B_n\}$, let $B'_i=B_{t_i}-ct_{i}$
\begin{eqnarray*}
    \int_A d\mathbb{P}^{x,c} & = & \int_{B'_1 \times B'_2\times \cdots \times B'_n} P_{t_1}(0,y_1)P_{t_2}(y_1,y_2)\cdots P_{t_n}(y_{n-1},y_n)dy_1dy_2\cdots d_{y_n}\\
    & = & \int_{B'_1 \times B'_2\times \cdots \times B'_n} \Pi \frac{1}{\sqrt{2\pi t_i}} e^{-\frac{y_1^2}{2t_i}}\times \cdots \times e^{-\frac{y_n^2}{2t_n}}dy_1dy_2\cdots d_{y_n}\\
    & = & \int_{B_1\times B_2\times \cdots \times B_n}  \Pi \frac{1}{\sqrt{2\pi t_i}} e^{-\frac{(x_1-ct_1)^2}{2t_1}} e^{-\frac{(x_2-ct_2)^2}{2t_2}}\times\cdots \times e^{-\frac{(x_n-ct_n)^2}{2t_n}}dx\\
    & = & \int_{B_1\times B_2\times \cdots \times B_n}  \Pi \frac{1}{\sqrt{2\pi t_i}}  e^{-\frac{(x_1)^2}{2t_1}} e^{-\frac{(x_2)^2}{2t_2}}\times\cdots \times e^{-\frac{(x_n)^2}{2t_n}}e^{-\frac{1}{2}c^2t+c \sum_{i=1}^n x_i}dx\\
    & = & \int_A  e^{-\frac{1}{2}c^2t+c(X_t-x)} d\mathbb{P}^{x,0}
\end{eqnarray*}
by $\pi-\lambda$ theorem, it holds for any $A\in \mathcal{F}_t$

\item 
\begin{eqnarray*}
    \mathbb{P}^{0,0}(S_t\leq x, X_t\leq  y)& = & \mathbb{P}^{0,0}(X_t\leq y)-\mathbb{P}^{0,0}(S_t> x , X_t\leq y)\\
    & = & \mathbb{P}^{0,0}(X_t\leq y)-\mathbb{P}^{0,0}(X_t\geq 2x-y )
\end{eqnarray*}
thus,  $\mathbb{P}^{0,c}=e^{cy-\frac{1}{2}c^2t}\cdot\sqrt{\frac{2t}{\pi}}(2x-y)e^{-\frac{(2x-y)^2}{2t}}$.
\end{enumerate}
























\noindent
\textbf{Problem 7}. (1) Let $B_t$ be a $d$-dimensional Brownian motion with $\{\mathcal{F}_t^B\}$ being its augemnted natural filtration.
\begin{enumerate}[(i)]
    \item Given $\theta \in \mathbb{R}^d$, define $e_{\theta}(x)=e^{i<x,\theta>}$ for $x\in \mathbb{R}^d$. Show that the process
    \[e_{\theta}(B_t)-1-\frac{1}{2}\int_0^t(\Delta e_{\theta})(B_s)ds,\quad t\geq 0\]
    is an $\{\mathcal{F}_t^B\}$-martingale
    \item Let $f$ be a smooth function on $\mathbb{R}^d$ with compact support. Taking as granted the fact that there exists a rapidly decreasing function $\phi$ on $\mathbb{R}^d$ (i.e. $\sup_{x\in \mathbb{R}^d}|x^{\alpha}{\partial}^{\beta}\phi(x)|< \infty \forall \alpha,\beta$), such that 
    \[f(x)=\int_{\mathbb{R}^d} e^{i<x,\theta>}\phi(\theta)d(\theta), \forall x\in \mathbb{R}^d\]
    show that the process 
    \[f(B_t)-f(0)-\frac{1}{2}\int_0^t(\Delta f)(B_s)ds, \quad t\geq 0.\]
    is an $\{\mathcal{F}_t^B\}$-martingale.
\end{enumerate}
(2) Let $X_t(\omega)\triangleq \omega_t$ be the coordinate process on $W^d$. Denote $\mathbb{P}_d^x$ on $(W^d,\mathcal{B}(W^d))$ as the law of a d-dimensional Brownian motion starting at $x\in \mathbb{R}^d$. 
\begin{enumerate}[(i)]
    \item Show that $f(x)\triangleq \log(|x|)$(in dimension $d=2$) and $f(x)=|x|^{2-d}$(in dimension $d\geq 3$) are harmonic on $\mathbb{R}^d/\{0\}$(i.e. $\Delta f(x)=0$ for every $x\neq 0$)
    \item Let $0<a <|x|<b$. Define $\tau_a$(respectively, $\tau_b$) to be the hitting time of the sphere $|y|=a$(respectively, $|y=b|$) be $X_t$. By choosing suitable $f$ in $(1)$, (ii), computer $\mathbb{P}^x_d(\tau_a<\tau_b)$ and $\mathbb{P}^x_d(\tau_a\leq \infty)$ in all dimendions $d\geq 2$.
    \item Le t$U$ be a non-empty, bounded open subset of $\mathbb{R}^d$. Define $\sigma=\sup\{t\geq 0:X_t\in U\}$. Show that $\mathbb{P}_d^0(\sigma=\infty)=1$ in dimension $d=2$, while $\mathbb{P}_d^0(\sigma<\infty)=1$ on dimension $d\geq 3$. Therefore, the Brownian motion is neighborhood-recurrent in dimension $d=2$ and neighborhood-transient in dimension $d\geq 3$.
    \item For every dimension $d\geq 2$, show that $\mathbb{P}_d^0 (\sigma_y\leq \infty)=0$ for every $y\in \mathbb{R}^d$, where $\sigma_y\triangleq \inf\{t>0:X_t=y\}$. Therefore, the Brownian motion is point-recurrent only in dimension one.
    Remark. It can be shown that in dimension $d=2$, with probability one, there exists $y\in \mathbb{R}^2$ such that the set $\{t\geq 0:X_t=y\}$ has cardinality of the continum.
\end{enumerate}

\begin{proof}
\begin{enumerate}[(1)]
\item 
\begin{enumerate}[(i)]
    \item by Ito formula,
 \[e_{\theta}(B_t)-1-\frac{1}{2}\int_{0}^t (\Delta e_{\theta})(B_s)ds=\sum_{i=1}^d\int_0^t i\theta_i e_{\theta}(B_s)dB^i_s\]
 since $B^i \in H_0^2$, and $|i\theta_ie_{\theta}(B_s)|=|\theta_i|^2$, we have $i\theta_ie_{\theta}(B_s)\in \mathcal{L}^2(B)$, thus $\int_0^t i\theta_i e_{\theta}(B_s)dB^i_s \in H_0^2$, 
 \[\sum_{i=1}^d\int_0^t i\theta_i e_{\theta}(B_s)\,dB^i_s\]
 is a martingale.
 \item by Ito formular 
 \[f(B_t)=f(0)+\sum_{i=1}^d\int_0^t\frac{\partial f}{\partial x_i}(B_s)dB^i_s+\frac{1}{2}\sum_{i,j=1}\frac{\partial^2 f}{\partial x^ix^j}(B_s)\,d \langle B^i_s,B^j_s\rangle\]
 thus 
 \[f(B_t)-f(0)-\frac{1}{2}\int_0^t(\Delta f)(B_s)ds=\sum_{i=1}^d\int_0^t \frac{\partial f}{\partial x_i}(B_s)\,dB_s\]
 since $\sup_{x\in \mathbb{R}^d}|x^{\alpha}\partial^{\beta}\phi(x)|<\infty$, we obtain 

\end{enumerate}
\item \begin{enumerate}[(i)]
    \item just by calculation.
    \item when $d=2$, 
    \[\mathbb{P}_d^x(\tau_a<\tau_b)=\frac{\log|b|-\log|x|}{\log|b|-\log|a|},\quad \mathbb{P}_d^x(\tau_a<\infty)=1\]
    when $d\geq 3$
    \[\mathbb{P}_d^x(\tau_a<\tau_b)=\frac{|b|^{2-d}-|x|^{2-d}}{|b|^{2-d}-|a|^{2-d}},\quad \mathbb{P}_d^x(\tau_a<\infty)=\frac{|x|^{2-d}} {|a|^{2-d}}\]
    
    
    \item  For $d=2$, Consider $B_r(x)\subset U$, Let $\tau_1=\inf \{t\geq 0: B_t\in B_r(x)\}$. $\tau_1$ is finite by (ii). Let $t_2=\inf\{t>t_1+1: B_t\in B_r(x)\}$, again $\tau_2$ is finite by strong Markov property. Repeating in this manner, we get $\tau_1<\tau_2<\cdots \tau_n<\cdots$, $\tau_n\uparrow \infty$  and $B_{\tau_n}\in B_r(x)\subset U$.
    \\
    For $d\geq 3$, consider $A_n=P(|B_t|>n, \text{for all} t >\tau_{n^3})$, by strong Markov property and (ii), 
    \[\mathbb{P}_d^0(A_n)=(\frac{1}{n^2})^{d-2}\]
    then by Borel-Cantelli lemma, there are only finite $A_n$ can occur, thus $|B_t|$ almost diverges. Therefore $\mathbb{P}_d^0(\sigma<\infty)=1$.
    
    
 
     
       \item It is equivalent to say that for Brownian Motion started at $y$, $\mathbb{P}^{y}_d(\sigma_0)=0$. Let $a\rightarrow 0$ in problem (ii), we obtain
       \[\mathbb{P}^{y}_d(\sigma_0<\infty)\leq \lim_{a\rightarrow 0}\mathbb{P}_d^y(\tau_a<\infty)=0\]

    \end{enumerate}

\end{enumerate}
\end{proof}































\end{document}