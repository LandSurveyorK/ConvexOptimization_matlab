% This is a sample LaTeX input file.  (Version of 12 August 2004.)
%
% A '%' character causes TeX to ignore all remaining text on the line,
% and is used for comments like this one.

\documentclass{article}     % Specifies the document class
\usepackage{amsmath}
\usepackage{amsthm}
\usepackage{graphicx}
\usepackage{mathtools}
\usepackage{amssymb}
\usepackage[utf8]{inputenc}
\usepackage[english]{babel}
\usepackage{enumerate}
\usepackage{float}
\newtheorem{theorem}{Theorem}[section]
\newtheorem{corollary}{Corollary}[theorem]
\newtheorem{lemma}[theorem]{Lemma}
\newtheorem*{remark}{Remark}


                             % The preamble begins here.
\title{Problem Sheet 1}  % Declares the document's title.
\author{WEI PENG}      % Declares the author's name.
\date{\today}      % Deleting this command produces today's date.

\newcommand{\ip}[2]{(#1, #2)}
                             % Defines \ip{arg1}{arg2} to mean
                             % (arg1, arg2).

%\newcommand{\ip}[2]{\langle #1 | #2\rangle}
                             % This is an alternative definition of
                             % \ip that is commented out.

\begin{document}             % End of preamble and beginning of text.
\maketitle    
% Produces the title

\section{Convex Sets}
(a,12pts) Closed sets and convex sets.
\begin{enumerate}[i.]
    \item Show that a polyhedron $\{x\in \mathbb{R}^n: Ax\leq b\}$, for some $A\in \mathbb{R}^{m\times n},b\in \mathbb{R}^m$, is both convex and closed.
    \item Show that if $S_i\subset \mathbb{R}^n,i\in I$ is a collection of convex sets, then their interaction $\bigcap_{i\in I}S_i$ is also convex. Show that the same statement holds if we replace "convex" with "closed".
    \item Given an example of a closed set in $\mathbb{R}^2$ whose convex hull is not closed.
    \item Let $A\in \mathbb{R}^{m\times n}$. Show that if $S\subset \mathbb{R}^m$ is convex so is $A^{-1}(S)=\{x:\in \mathbb{R}^{n}: Ax=S\}$, which is called the preimage of $S$ under the maping $A;\mathbb{R}^{n}\rightarrow \mathbb{R}^{m}$. Show that the same statement holds if we replace "convex" with "closed".
    \item Let $A\in \mathbb{R}^{m\times n}$. Show that if $S\subset \mathbb{R}^{n}$ is convex then so is $A(S)=\{Ax:x\in S\}$, called the image of $S$ under $A$.
    \item Give an example of a matrix $A\in \mathbb{R}^{m\times n}$ and a set $S\subset \mathbb{R}^n$ that is closed and convex but such that $A(S)$ is not closed.
\end{enumerate}

\begin{proof}
\begin{enumerate}[i.]
\item  $A[\theta x +(1-\theta)y]= \theta Ax +(1-\theta) Ay\leq b$. Since $A:x\rightarrow Ax$ is is continuous mapping. so the preimage of the closed set $(-\infty,b]$ is absolutely closed.

\item  Suppose $x,y\in \bigcap_{i\in I}S_i$, since  $S_i$ is convex, we have $\theta x +(1-\theta)y \in S_i, \forall i\in I$. If replace "convex" by "closed", it is also true as the property of a topology space.
\item  $C=\{(x,y):y=x^2,x^2-y^2=1,x>0\}\bigcup \{(0,0)\}$.

\item $\theta A^{-1}x + (1-\theta)A^{-1}y = A^{-1}[\theta x +(1-\theta)y]$. If replaces "convex" with "closed", it is also true since the preimage of a closed set by a continuous function is still closed.

\item $\theta Ax +(1-\theta) Ay=A[\theta x +(1-\theta)y]$.
\item $S=\{(x,y):x>0,y\geq 1/x\}$, $A: (x,y)\mapsto x$.
\end{enumerate}






\end{proof}




(b,4pts) The following is an important property of polyhedra:


$P \subset \mathbb{R}^{m+n}$ is a polyhedra $\Rightarrow$ $\{x\in \mathbb{R}^n: (x,y)\in P$~for~some~$y\in\mathbb{R}^m\}$ is a polyhedra.~~~~~~~~~~~~~~~~~(1)


(Bonus:prove this property)
\begin{enumerate}[i.]
    \item Use the above property about polyhedra to conclude that if $A \in \mathbb{R}^{m\times n}$ and $P\subset \mathbb{R}^{n}$ is a polyhedra then $A(P)$ is a polyhedra.
    \item Given an example to show that (1) is no longer true if we replace "polyhedra" with "close and convex set".
\end{enumerate}

\begin{proof}
\begin{enumerate}[i.]

    \item $\{Ax;x\in P\}=\{y:(x,y\in P^*\}$, 
   where $P^*=\{(x,y): x\in P \& Ax=y\}$
    \item same as (a) iv.

 \end{enumerate}
\end{proof}

(c,4pts) The following is a "strict" variant of the Separating Hyperplane Theorem: if $C,D\subset \mathbb{R}$ are disjoint closed and convex, and (say) $D$ is bounded, then there exists $a\in  \mathbb{R}, b\in  \mathbb{R}$ with $a\neq 0$ such that $a^Tx>b$ for all $x\in C$(i.e. the hyperplane$\{x\in  \mathbb{R}:a^Tx=b\}$ strictly separeates $C,D$. Use this to prove Farkas' Lemma: give $A\in  \mathbb{R}^{m\times n}$, $b\in \mathbb{R}^m$, exactly one of the following is true:

\begin{itemize}
    \item $\exists x\in  \mathbb{R}^{n}$ such that $Ax=b,x\geq 0$.   (1)
    \item $\exists y \in  \mathbb{R}^m$ such that $A^Ty\geq 0, y^Tb<0$.    (2)
\end{itemize}

\begin{proof}
On one hand, suppose $Ax^*=b, x\geq 0$, then $\forall y$, $0>y^Tb=y^T(Ax^*)=(x^*)^T(A^Ty)$, then $A^Ty$ can not be $\geq 0$.

On the other hand, if there is no feasible solution of (1), Consider $\{Ax:\geq 0\}$ and $\{b\}$, they are two disjoint closed and convex sets, and \{b\} is bounded. By the Separating Hyperplane Theorem above, There exists $y$, such that $y^Tb<0$, and $y^T(Ax)=x^T(Ay)\geq 0, \forall x\geq 0$, which implies $y$ is a feasible value of (2).


\end{proof}


\section{Convex Function}

(a,6pts) Prove that $f(x,y)=|xy|+a(x^2+y^2)$ is convex if and only if $a\geq 1/2$. Also prove that it is strongly convex if $a>1/2$. (Bonus: product 3d plots of $f$, and each $a\in \{0,1/4,1/2,3/4\}$.)

\begin{proof}
\begin{itemize}
    
    \item 
If $a<1/2$, consider points $(1,0)$ and $(0,1)$, 
\[f((\theta,1-\theta))=\theta (1-\theta)+a(\theta^2+(1-\theta)^2)>a=\theta f((1,0))+(1-\theta)f((0,1))\]

If $a\geq 1/2$, it is sufficient to consider $a=1/2$, since $\frac{m}{2}(x^2+y^2)$ is a convex function for any $m>0$.
\item 
if $a>1/2$, 
 $$f((x,y))=|xy|+\frac{1}{2}(x^2+y^2)+\frac{2a-1}{2}(x^2+y^2)$$ 
is strictly convex function since $\frac{1}{2}(x^2+y^2)$ is convex.

if $a\leq 1/2$, there is no $m>0$, such that 
 $$f((x,y))-\frac{m}{2}(x^2+y^2)=|xy|+(a-\frac{m}{2})(x^2+y^2)$$ 
is convex since $(a-\frac{m}{2})<1/2$.

\end{itemize}
\end{proof}

(b,6pts) In each case below specify whether the function is strongly convex, strictly convex, or nonconvex, and give a brief justification.
\begin{enumerate}
    \item The \emph{logarithmic barrier}, $f: \mathbb{R}_{++}\rightarrow  \mathbb{R}$:
\[ f(x) = -\sum_{i=1}^{n} \log(x_i).  \]
    \item The \emph{entropy function}, $f:\{x\in  \mathbb{R}_+^n: \sum_{i=1}^{n}x_i=1\}\rightarrow  \mathbb{R}$ defined as:
    \[ 
    f(x)=\left\{ 
    \begin{array}{cc}
     -\sum_{i=1}^{n} x_i\log(x_i) & \quad if ~x>0\\
     0 &\quad otherwise.
     \end{array}  \right.
     \]
\end{enumerate}

\begin{proof}
\begin{enumerate}
\item $f_i(x):x\mapsto -\log(x_i)$ is strictly convex, and the sum of strictly convex functions is still strictly convex, whcih implies f(x) is strictly convex.(It is not hard to see that $f(x)$ is not strongly convex, since $f(x)-\frac{m}{2}||x||_2^2=\sum [-\log(x_i)-\frac{m}{2}(x_i^2)]$, for example, just consider $\{(x_1,0,\cdots,0)\}$).

\item Similar to 1. $f(x)$ is strictly convex but not strongly convex.
\end{enumerate}
\end{proof}

(c,4pts) Let $f$ be a twice differentiable, with $dom(f)$ convex. Prove that $f$ is convex if and only if 
\[ (\nabla f(x)-\nabla f(y))^T(x-y)\geq 0
,\]
for all x,y. This property is called \emph{monotonicity} of the gradient $\nabla f$.

\begin{proof}
"$\Rightarrow$": if $f(x)$ is convex, then
\[\left\{
\begin{array}{c}
f(y)-f(x)\geq \nabla f(X)^T(y-x)\\
f(y)-f(x)\geq \nabla f(X)^T(y-x)
\end{array}\right.
\Rightarrow (\nabla f(x)-\nabla f(y))^T(x-y)\geq 0\]
"$\Leftarrow$": Suppose  $(\nabla f(x)-\nabla f(y))^T(x-y)\geq 0$, since $\nabla f(y)=\nabla f(x)+\nabla^2 f(x)(y-x)+o(||x-y||^2)$, which implies
\[\nabla^2 f(x)\geq 0,~~\forall x\]
thus $f(x)$ is convex.


\end{proof}

\section{Lipschitz gradient and strong convexity}
Let $f$ be convex and twice differentiable.

(a,8pts) Show that the following statements are equivalent.
\begin{enumerate}[i.]
    \item $\nabla f $ is Lipchitz with constant $L$;
    \item $ (\nabla f(x)-\nabla f(y))^T(x-y)\leq L||x-y||_2^2$;
    \item $\nabla^2 f(x) \preccurlyeq LI$ for all $x$;
    \item $f(y)\leq f(x) +\nabla f(x)^T(y-x)+\frac{L}{2}||y-x||^2_2$ for all $x,y$.
    \end{enumerate}

\begin{proof}
\begin{itemize}
    \item  i $\Rightarrow$ ii: $ (\nabla f(x)-\nabla f(y))^T(x-y)\leq ||\nabla f(x)-\nabla f(y)||||x-y||\leq L||x-y||_2^2$.
    \item  ii $\Rightarrow$ iii:
    \begin{eqnarray*} 
        (y-x)^T\nabla^2 f(X)(y-x)& = &(y-x)^T(\nabla f(x)-\nabla f(y))^T(x-y)-o(||y-x||_2^2)\\
        & = & \leq L|y-x|||^2-o(||y-x||_2^2)
    \end{eqnarray*}
    which implies iii.
     
    \item iii $\Rightarrow$ iv:
\begin{eqnarray*} 
         f(y) & = & f(x) +\nabla f(x)^T(y-x)+\frac{1}{2}\nabla^2 f(\widetilde{x})(y-x)\\
       & \leq &  f(x) +\nabla f(x)^T(y-x)+\frac{L}{2}||y-x||^2_2
    \end{eqnarray*}

    \end{itemize}
\end{proof}

(b,8pts) Show that the following statements are equivalent.
\begin{enumerate}[i]
    \item $f$ is strongly convex with constant $m$.
    \item $ (\nabla f(x)-\nabla f(y))^T(x-y)\geq m||x-y||_2^2$ for all $x,y$.
    \item $\nabla^2 f(x)\succcurlyeq m I$ for all $x$;
    \item $f(y)\geq f(x) +\nabla f(x)^T(y-x)+\frac{m}{2}||y-x||^2_2$ for all $x,y$.
\end{enumerate}

\begin{proof}
\begin{itemize}
 \item  i $\Rightarrow$ ii: if $f(x)$ is strongly convex, which means for any $m>0$, $f(x)+\frac{m}{2}||x||_262$ is convex. Then similar to i $\Rightarrow$ ii above, we will done.
 \item  ii $\Rightarrow$ iii: similar to ii $\Rightarrow$ iii above.
 \item iii $\Rightarrow$ iv: similar to ii $\Rightarrow$ iii above.
 \item  iv $\Rightarrow$ i:
    \begin{eqnarray*} 
         iv  & \Rightarrow  &f(y)-\frac{m}{2}||y||_2^2\geq f(x)-\frac{m}{2}||x||_2^2+\nabla(f(x)-\frac{m}{2}||x||_2^2)(y-x)\\
        & \Rightarrow & f(x)-\frac{m}{2}||x||_2^2)~is~convex
    \end{eqnarray*}

    \end{itemize}
\end{proof}

\section{Solving optimization problems with CVX}
(a) Using CVX, we will solve the 2d fused lasso problem discusses in Lecture1:
\[\min_{\theta}\frac{1}{2}\sum_{i=1}^{n}(y_i-\theta_i)^2+\lambda\sum_{\{i,j\}\in E}|\theta_i-\theta_j|\]

\begin{figure}[hb]
  \centering
  \includegraphics[width=4in]{xy}
  \caption[]
   { original image vs. solution image}
\end{figure}
The object value is 182.799, because $\lambda = 1$, adds great penalty on the difference of neighboring pixels. Thus it forces the solution to be the same in neighborhood.


(b)

\begin{center}
\begin{tabular}{c|c|c|c|c|c|c|c|c|c}
 $\lambda$ & 0 & 1 & 2 & 3 & 4 & 5 & 6 & 7 &8\\
\hline
optval &78.8310 &77.2784 & 72.8403 & 61.8807  & 47.7741 & 34.4890 & 23.6211 & 15.4753 & 9.7419
\end{tabular}
\end{center}

\begin{figure}[H]
  \centering
\includegraphics[width=4in]{0}
  \caption[]
   { original image vs. solution image($\lambda=0$)}
\end{figure}

\begin{figure}[H]
  \centering
\includegraphics[width=4in]{1}
  \caption[]
   { original image vs. solution image($\lambda=1$)}
\end{figure}
\begin{figure}[H]
  \centering
\includegraphics[width=4in]{2}
  \caption[]
   { original image vs. solution image($\lambda=2$)}
\end{figure}
\begin{figure}[H]
  \centering
\includegraphics[width=4in]{3}
  \caption[]
   { original image vs. solution image($\lambda=3$)}
\end{figure}
\begin{figure}[H]
  \centering
\includegraphics[width=4in]{4}
  \caption[]
   { original image vs. solution image($\lambda=4$)}
\end{figure}
\begin{figure}[hb]
  \centering
\includegraphics[width=4in]{5}
  \caption[]
   { original image vs. solution image($\lambda=5$)}
\end{figure}
\begin{figure}[H]
  \centering
\includegraphics[width=4in]{6}
  \caption[]
   { original image vs. solution image($\lambda=6$)}
\end{figure}
\begin{figure}[H]
  \centering
\includegraphics[width=4in]{7}
  \caption[]
   { original image vs. solution image($\lambda=7$)}
\end{figure}

\begin{figure}[H]
  \centering
\includegraphics[width=4in]{8}
  \caption[]
   { original image vs. solution image($\lambda=8$)}
\end{figure}


The points in histogram are centered around 0 first, then spreading as  $\lambda$ increading.
\\

(b,8pts) Writing problems in DCP form.
\begin{enumerate}
    \item 
\end{enumerate}




























\end{document}