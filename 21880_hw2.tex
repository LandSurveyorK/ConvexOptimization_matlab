
\documentclass{article}     % Specifies the document class
\usepackage{amsmath}
\usepackage{amsthm}
\usepackage{mathtools}
\usepackage{amssymb}
\usepackage[utf8]{inputenc}
\usepackage[english]{babel}
\newtheorem{theorem}{Theorem}[section]
\newtheorem{corollary}{Corollary}[theorem]
\newtheorem{lemma}[theorem]{Lemma}
\newtheorem*{remark}{Remark}
\usepackage{enumerate}
\usepackage{float}
                             % The preamble begins here.
\title{Problem Sheet 5}  % Declares the document's title.
\author{WEI PENG}      % Declares the author's name.
\date{\today}      % Deleting this command produces today's date.

\newcommand{\ip}[2]{(#1, #2)}
                             % Defines \ip{arg1}{arg2} to mean
                             % (arg1, arg2).

%\newcommand{\ip}[2]{\langle #1 | #2\rangle}
                             % This is an alternative definition of
                             % \ip that is commented out.

\begin{document}             % End of preamble and beginning of text.
\maketitle             % Produces the title.

\emph{You are incourage to discuss with your classmates whenever you find it helpful}

\noindent


\textbf{Problem1}. (1) Suppose that $\{X_t:\mathcal{F}_t:t\geq 0\}$ is a right continuous supermartingale and $\tau$ is an $\{\mathcal{F}_t\}$-stopping time. Show that the stopped process $X_t^{\tau} \triangleq X_{\tau \wedge t}$ is a supermartingale both with respect to the filtrations $\{\mathcal{F}_t:t\geq 0\}$ and $\{\mathcal{F}_{\tau \wedge t}:t\geq 0\}$.

(2) Let $\{X_t:t\geq 0\}$ be an $\{\mathcal{F}_t\}$-adapted and right continuous stochastic process. Suppose that for any bounded stopping time $\sigma \leq \tau$, $X_{\sigma}\leq X_{\tau}$. Show that $\{X_t,\mathcal{F}_t\}$ is a submartingale.

\begin{proof}
(1) By bounded optional sampling theorem.
\[E[X_{\tau \wedge t}|\mathcal{F}_{\tau \wedge s}]\leq X_{\tau \wedge s}\]
since $X_{\tau \wedge t}\in \mathcal{F}_{\tau}$,
\[E[X_{\tau \wedge t}|\mathcal{F}_s]=E[E[X_{\tau \wedge t}|\mathcal{F}_{\tau}]|\mathcal{F}_s]=E[X_{\tau \wedge t}|\mathcal{F}_{\tau\wedge s}]\leq X_{\tau \wedge s}\]
Measurability and integrability are staightforward.

(2) For any $A\in \mathcal{F}_s$, consider the following stopping time
\[\tau=\begin{cases}
t, & \omega \in A\\
s, & \omega \in A^c
\end{cases}\]
by O.S.T. $E[X_{\tau}]\geq E[X_s]$, which means $E[X_t1_A]\geq E[X_s1_A]$.
\end{proof}

\noindent
\textbf{Problem2}. Let $(\Omega,\mathcal{F},\mathbb{P};\{\mathcal{F}_t:t\geq 0\})$ be a filtered probability space which satisfies the ususal conditions. Suppose that $\mathbb{Q}$ is another probability measure on $(\Omega,\mathcal{F},\mathbb{P})$ satisfying that $\mathbb{Q}\ll \mathbb{P}$ on $\mathcal{F}_t$ for every $t\geq 0$.

\begin{enumerate}[(1)]
    \item Let $M_t$ be a version of $d \mathbb{Q}/d \mathbb{P}$. Show that $\{M_t,\mathcal{F}_t\}$ is a martingale.
    \item Take a cadlag modification of $M_t$ and still denote it by $M_t$ for simplicity. Show that $\{M_t\}$ is uniformly integrable if and only if $\mathbb{Q}\leq \mathbb{P}$ on $\mathcal{F}_{\infty}$, In this case, we have
    \begin{enumerate}[(i)]
        \item $M_{\infty}\triangleq \lim_{t\rightarrow \infty} M_t= d \mathbb{Q}/d \mathbb{P}$ on $\mathcal{F}_{\infty}$.
        \item for every $\{\mathcal{F}_t\}$-stopping time $\tau$, $\mathbb{Q}\leq \mathbb{P}$ on $\mathcal{F}_{\tau}$ and $M_{\tau}=d\mathbb{Q}/\mathbb{P}$ on $\mathcal{F}_{\tau}$.
        \end{enumerate}
    \end{enumerate}
\begin{proof}
(1) Since $M_t$ is a version of $d \mathbb{Q}/d \mathbb{P}$, then $M_t$ is $\mathcal{F}_t$-measurable, and for any $A\in \mathcal{F}_s\subset\mathcal{F}_t$,
\[\int_{A}M_td\mathbb{P}=\mathbb{Q}(A)=\int_{A}M_sd\mathbb{P}\]
By the definition of conditional expectation, $\{M_t,\mathcal{F}_t\}$ is a martingale.

(2) By the definition of $M_t$, it is non-negative, thus
\[E[M_t]=\int M_t d\mathbb{P}=\mathbb{Q}(\Omega)=1~~~~~\quad (*)\]
If $M_t$ is U.I., for any $A\in \mathcal{F}_{\infty}$ satisfying $\mathbb{P}(A)=0$, then $\exists t>0$, $\forall s\geq t$, $A\in \mathcal{F}_s$
\[Q(A)=\int_A M_sd\mathbb{P}=0, \quad  (by ~uniform~equicontinuity) \]
on the other hand, if $\mathbb{Q}\ll \mathbb{P}$ on $\mathcal{F}_{\infty}$, let $M^*$ be a version of $d \mathbb{Q}/d \mathbb{P}$ on $\mathcal{F}_{\infty}$.
\[\sup_t \int_{M_t\geq \lambda}M_td\mathbb{P}=\sup_t \mathbb{Q}(M_t>\lambda)=\sup_t\int_{M_t>\lambda}M^*d\mathbb{P}\rightarrow 0\quad (\lambda\rightarrow 0)\]
so $\{M_t\}$ is U.I.
\begin{enumerate}[(i)]
    \item By (*) and U.I., there exists $M_\infty$, such that $M_t \rightarrow M_{\infty}$ a.s.and in $L^1$,   $M_{\infty}\in \mathcal{F}_{\infty}$, and for any $A\in \mathcal{F}_{\infty}$, $\exists t\geq 0$, such that for any $s\geq t$, $A\in \mathcal{F}_s$.
    \[\mathbb{Q}(A)=\int_A M_sd\mathbb{P}\]
    by D.C.T. we have $\mathbb{Q}(A)=\int_A M_{\infty}d\mathbb{P}$, which means $M_{\infty}$ is a version of $d \mathbb{Q}/d \mathbb{P}$ on $\mathcal{F}_{\infty}$.
    \item  Since $\mathcal{F}_{\tau}\subset \mathcal{F}_{\infty}$ and $\mathbb{Q}\ll \mathbb{P}$ on $\mathcal{F}_{\infty}$, so $\mathbb{Q}\ll \mathbb{P}$ on $\mathcal{F}_{\tau}$. and snce $E[M_{\infty}|\mathcal{F}_{\tau}]=M_{\tau}$, we have 
    \[Q(A)=\int_{A}M_{\infty}d\mathbb{P}=\int_{A}M_{\tau}d\mathbb{P}\]
    thus, $M_{\tau}=d \mathbb{Q}/d\mathbb{P}$ on $\mathcal{F}_{\tau}$. 

\end{enumerate}


\end{proof}


\noindent 
\textbf{Problem3}. Let $\{X_t, \mathcal{F}_t\}$ be a right continuous martingale which is bounded in $L^p$ for some $p>1$(i.e. $\sup_{0\leq t<\infty}E[|X_t|^p]\leq M<\infty$). Show that $X_t$ converges to some $X_{\infty}$ almost surely and in $L^p$.


\begin{proof}
\[E|X_t|\leq (E[|X_t|^p])^{\frac{1}{p}}\]
so $E[|X_t|]$ is uniformly bounded, by Doob's convergence theorem, $X_t\rightarrow X_{\infty}$, a.s.
Moreover $X_t$ is U.I. since 
\[\int_{|X_t|>\lambda} |X_t|\leq \left(E[|X^t|^p]\right)^{\frac{1}{p}}\left(P(|X_t|>\lambda)\right)^{\frac{1}{q}}\leq M\left(P(|X_t|>\lambda)\right)^{\frac{1}{q}}\rightarrow 0\quad \lambda\rightarrow \infty\]
Thus $X_t\rightarrow X_{\infty}$ a.s and in $L^1$, and by Doob's $L^p$ inequality,
\[||\sup_{0\leq t\leq T}X_t||_p\leq \frac{p}{p-1}||X_T|_p\leq M^{\frac{1}{p}}\]
and since  $|X_t-X_{\infty}|^p\leq (2\sup_{0\leq t\leq \infty}X_t)^p$, by D.C.T, $X_t\rightarrow X_{\infty}$ a.s. and in $L^p$.


\end{proof}

\noindent
\textbf{Problem4}. (1) Show that $\log t\leq t/e$ for every $t>0$, and conclude that 
\[a\log^+t\leq a \log^+a +\frac{b}{e}\]
for every $a,b>0$, where $\log^+t=\max\{0,\log t\}$(t>0).

(2) Suppose that $\{X_t,\mathcal{F}_t:t\geq 0\}$ is a non-negative and right continuous submartingale. Let $\rho: [0,\infty)\rightarrow R$ be an increasing and right continuous function with $\rho(0)=0$. Show that 
\[E[\rho(X^*_T)]\leq E[X_T\int_0^{X^*_T}\lambda^{-1}d\rho(\lambda)], \quad \forall T>0\],
where $X^*_T\triangleq \sup_{[0,T]}X_t$.

(3) By choosing $\rho(t)=(t-1)^+(t\geq 0)$, show that 
\[E[X^*_T]\leq \frac{e}{e-1}(1+E[X_T\log+X_T]),\quad \forall T>0\].

\begin{proof}
(1) trivial.

(2) \begin{eqnarray*}
E[\rho(X^*_T)]& = &\int_0^{\infty}P(\rho(X^*_T)>\lambda)d\lambda\\
& \leq & \int_0^{\infty}\lambda^{-1}E[X_T1_{  \{\rho(X^*_T)>\lambda\}  }]d\lambda\\
& = &  E[X_T\int_0^{X^*_T}\lambda^{-1}d\rho(\lambda)]
\end{eqnarray*}

(3)\begin{eqnarray*}
E[X^*_T] & = & E[X^*_T1_{\{X^*_T>1\}}]+E[X^*_T1_{\{X^*_T\leq 1\}}]\\
& \leq & E[\rho(X^*_T)]+1\\
& \leq & E[X_T\log^+X_T^*]+1\\
& \leq & E[X_T\log^+ X_T]+E[X^*_T]/e +1
\end{eqnarray*}
which implies 
\[E[X^*_T]\leq \frac{e}{e-1}(1+E[X_T\log+X_T]),\quad \forall T>0\].

\end{proof}

\noindent 
\textbf{Problem5}. Suppose that $\{X_t,\mathcal{F}_t:t\geq 0\}$ is a continuous martingale and 
\[\sup_{t\geq 0}X_t(\omega)=\infty\quad \inf_{t\geq 0}X_t(\omega)=-\infty, \quad \omega\in \Omega\]
Define $\tau_0=0$, and $\tau_n=\inf\{t>\tau_{n-1}:|X_t-X_{\tau_{n-1}}|=1\}(n\geq 1)$. Show that $\tau_n$ are finite $\{\mathcal{F}_t\}$-stopping time. What is the distribution of the random sequence $\{X_{\tau_n}:n\geq 1\}$.

\begin{proof}
Suppose $\tau_{n-1}$ is a finite stopping time. For any $\omega$, since it has continuous path, it is bounded in $[0,\tau_{n-1}(\omega)]$, and since $\sup_{t\geq 0}X_t(\omega)=\infty$ and  $\inf_{t\geq 0}X_t(\omega)=-\infty$, so $\tau_n(\omega)=\inf\{t>\tau_{n-1}(\omega):|X_t-X_{\tau_{n-1}}|=1\}<\infty$. by O.S.T.
\[\begin{cases}
P\left(E[X_{\tau_n}]|\mathcal{F}_{\tau_{n-1}}]=X_{\tau_{n-1}}+1\right)=1/2\\
P\left( E[X_{\tau_n}]|\mathcal{F}_{\tau_{n-1}}]=X_{\tau_{n-1}}-1\right)=1/2
\end{cases}\]

Let $\xi_n$ be i.i.d. uniformly distribution on $\{-1,1\}$, then we have $$E[[X_{\tau_n}|X_{\tau_{n-1}}]=E[E[X_{\tau_n}]|\mathcal{F}_{\tau_{n-1}}]|X_{\tau_{n-1}}]=X_{\tau_{n-1}}+\xi_n$$
By induction, we have $X_{\tau_n}=X_0+\sum_{i=1}^n \xi_n$, which is a random walk.

\end{proof}


\noindent
\textbf{Problem6}. Let $\{X_t,\mathcal{F}_t\}$ be a continuous martingale which is uniformly integrable. Suppose there exists a constant $M_X>0$ such that 
\[E[|X_{\infty}-X_{\tau}|\mathcal{F}_{\tau}]\leq M_X\quad a.s.\]
for every $\{\mathcal{F}_t\}$-stopping time $\tau$, where $X_{\infty}=\lim_{t\rightarrow \infty} X_t$ which  exists amost surely and in $L^1$ according to uniformly integrability. Let $X^*=\sup_{t\geq 0}|X_t|$.
\begin{enumerate}[(1)]
    \item  Show that  for every $\lambda,\mu>0$,
    \[P(X^*>\lambda+\mu)\leq \frac{M_X}{\mu}P(X^*>\lambda)\]
    \item By using the result of (1), show that 
    \[P(X^*>\lambda)\leq e^{2-\frac{\lambda}{e\cdot M_X}},\quad \forall \lambda>0\]
    in particular, $\exp{\alpha X^*}$ is integrable when $ 0< \alpha<(eM-X)^{-1}$, which also implies that $X^*\in L^p$ for every $p\geq 1$.

\end{enumerate}

\begin{proof}
(1) $\{X_t,\mathcal{F}_t;0\leq t\leq \infty\}$ is a martingale with last element, Let $\tau = \inf \{t\geq 0:X_t\geq \lambda+\mu\}$, then by O.S.T.
\[E[(X_{\infty}-\lambda) 1_{\tau<\infty}]=E[(X_{\tau}-\lambda) 1_{\tau<\infty}]=E[\mu 1_{\tau <\infty}]=\mu P(\tau<\infty)\]

let $\sigma=\inf\{t\leq 0: X_t\geq \lambda\}$
\begin{eqnarray*}
\mu P(X^*>\lambda+\mu) & = & \mu P(\tau<\infty)\\ 
& \leq &  E[|X_{\infty}-\lambda)|1_{\tau<\infty}]\\ 
& \leq &  E[|X_{\infty}-\lambda)|1_{\sigma<\infty}]\\ 
& =&  E[|X_{\infty}-X_{\sigma})|1_{\sigma<\infty}]\\
& \leq & M_XP(\tau\leq \infty)\\ 
& = & M_X P(X^*>\lambda)
\end{eqnarray*}





(2)(????? fail to prove ) By observating, there exists some $\lambda^*>0$, the inequality holds when $\lambda \leq \lambda^* $.  
for $\lambda>\lambda^*$, the following inequality will hold
\begin{eqnarray*}
P(X^*>\lambda) & \leq &\frac{M_X}{\lambda-\lambda^*}P(X^*>\lambda^*)\\
& \leq  & \frac{M_X}{\lambda-\lambda^*}e^{2-\frac{\lambda^*}{e.M_x}}\\
& \leq &(? fail )~ e^{2-\frac{\lambda}{e\cdot M_X}}
\end{eqnarray*}














\end{proof}

\noindent
\textbf{Problem7}. Let $\{X_t,\mathcal{F}_t:t\geq 0\}$ be a cadlag submartingale over a filtered probability space which satisfies the usual conditons.

\begin{enumerate}[(i)]
    \item  Suppose that $X_t$ is non-negative, show that $X_t$ is of class (DL). Suppose further that $X_t$ is continuous, show that $X_t$ is regular.
    \item  Suppose that $X_t$ is uniformly integrable. Show that $X_t$ is of class (D) in the sense that $\{X_{\tau}:\tau \in \mathcal{S}\}$ is iniformly integrable, where $\mathcal{S}$ is the set of finite $\{\mathcal{F}_t\}$-stopping times. Moreover, $A_{\infty} \triangleq \lim_{t\rightarrow \infty }A_t$ is integrable, where $A_t$ is the natural increasing process in the Doob-Meyer decomposition of $M_t$.
\end{enumerate}

\begin{proof}
\begin{enumerate}[(i)]
    \item First, let's show that $E[X_T|\mathcal{F}_\tau]$ is U.I.
    \begin{eqnarray*}
        E[|E[X_T|\mathcal{F}_\tau]|1_{|E[X_T|\mathcal{F}_\tau]|>\lambda}] &\leq & E[E[|X_T||\mathcal{F}_\tau]1_{E[|X_T||\mathcal{F}_\tau]>\lambda}]\\
        & = & E[|X_T|1_{E[|X_T||\mathcal{F}_\tau]>\lambda}] \\
        & = & E[|X_T|1_{E[|X_T||\mathcal{F}_\tau]>\lambda};|X_T|>\sqrt{\lambda}]+\\
        &~~~& E[|X_T|1_{E[|X_T||\mathcal{F}_\tau]>\lambda};|X_T|\leq\sqrt{\lambda}]\\ 
        & \leq &  E[|X_T||X_T|>\sqrt{\lambda}]+ \sqrt{\lambda}E[1_{E[|X_T||\mathcal{F}_\tau]>\lambda}]\\ 
        & \leq & E[|X_T||X_T|>\sqrt{\lambda}]+ \sqrt{\lambda}\frac{E[|X_T|]}{\lambda}\\ 
        & \rightarrow & 0 \quad (\lambda \rightarrow \infty)    
    \end{eqnarray*}
    Since  $0\leq X_{\tau}\leq E[X_T|\mathcal{F}_{\tau}]$, which means that $X_{\tau}$ is controlled by $E[X_T|\mathcal{F}_{\tau}]$ which is U.I., we obtain that $X_{\tau}$ is U.I.
    
    If $X_t$ is continuous, $\tau_{n} \uparrow \tau $, since $EX_{\tau}\leq E[X_T]$, So by D.C.T. 
    \[\lim_{n\rightarrow \infty}E[X_{\tau_{n}}]=E{X_{\tau}}\]
    $X_t$ is regular.
   
    \item Since $X_t$ is U.I., we have $X_t\rightarrow X_{\infty}$ a.s. and in $L^1$. Then $\{X_{t}:0\leq t\leq \infty\}$ is a non-negative supermartingale with last element. By O.S.T. \[E[X_{\infty}|\mathcal{F_{\tau}}]\geq X_{\tau}\geq 0\]
    again, since $X_{\tau}$ is controlled by some conditional expectation, thus $X_{\tau}$ is U.I.
    
    \begin{eqnarray*}
        E[A_{\infty}]& \leq& \lim\inf_tE[A_t]\quad (Fatou~Lemma)\\ 
        & = & \lim\inf_{t}E[X_t-M_t]\\
        &=& \lim\inf E[X_t]-E[M_0]\\
        &\leq & \sup_t E[X_t]-E[M_0]\\
        &< & \infty
        \end{eqnarray*}
  So, $A_{\infty}$ is integrable.
    \end{enumerate}

\end{proof}











































\end{document}